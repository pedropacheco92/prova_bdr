\documentclass{article}
\usepackage[utf8]{inputenc}

\title{Documentação Locadora de Carros}
\author{Pedro Henrique Pacheco}
\date{Junho 2018}

\begin{document}
\maketitle

O objetivo desde projeto é criar uma locadora de carros,
porém com a implementação de apenas o cadastro de carros.

Esse projeto com criado utilizando \textbf{Java} no \textit{back-end} com \textit{Spring Boot}
e \textbf{Angular 6} no \textit{front-end}.

E para documentação foi utilizando alguns conceitos da técnica Caso de Uso 2.0,
a qual vem para estender o Caso de Uso afim de torna-lo uma técnica mais ágil.

\section{História de usuário}
Como usuário do sistema de locadora, eu quero ver todos os carros que atualmente temos cadastrados,
bem como poder incluir novos carros, edita-los ou até mesmo excluí-los.

\section{Gerenciamento de carros}
\textbf{Ator:} usuário do sistema.
\subsection{Pré-condição}
O usuário deverá possuir algum carro cadastrado.

\subsection{Fluxo básico}
\begin{enumerate}
    \item Caso começa com o usuário na listagem de carros.
    \item A tela contém uma lista com paginação e um botão para adicionar um carro.
    \item Ao clicar em adicionar um carro ele deve navegar para a tela contendo o formulário do cadastro do carro, com as opções de \textbf{Criar} e \textbf{Cancelar}
    \item Ao Salvar, deve mostrar uma notificação de sucesso e retomar a listagem de carros.    
\end{enumerate}

\subsection{Fluxos alternativos}
\subsubsection{Excluir carro}
\begin{enumerate}
    \item Na listagem de carros, deve-se clicar na ação de exclusão.
    \item Após a confirmação da exclusão deve continuar na listagem de carros.
\end{enumerate}

\subsubsection{Edição do carro}
\begin{enumerate}
    \item Na listagem de carros, deve-se clicar na ação de edição.
    \item O sistema navega para a tela de edição do carro com as opções \textbf{Salvar} e \textbf{Cancelar}
    \item Ao Salvar, deve mostrar uma notificação de sucesso e retomar a listagem de carros.
\end{enumerate}

\subsubsection{Visualização do carro}
\begin{enumerate}
    \item Na listagem de carros, deve-se clicar na ação de visualizar.
    \item O sistema mostra as informações do carro em formato de modal.
\end{enumerate}

\subsection{Pós-condição}
Ao realizar alguma ação sobre o carro, a listagem sempre deve ser atualizada.

\section{Use Case Slices}
\subsection{Slice 01}
Fluxo básico

\subsection{Slice 02}
Fluxos alternativos: exclusão, edição e visualização;

\end{document}
